In any medical organisation, it is essential to keep a good organisation of the data about individual patients. Besides the usual requirements for this data, in terms of being easily accessible and well structured, new legislations such as GDPR put additional requirements in terms of the organisation, ownership, access and communication of this data. The data is owned by the patient themselves and they need to be able to have full access to it, while the other parties (general practicioners, specialists, insurers) must have access to only the parts of the data that are releveant to diagnostics, treatment and insurance. Currently, there is no unified way to represent the patient data across different organisations. Each individual organisation (e.g.~National Health Service in the UK) might have their own way of organising the data. Quite frequently, this data is not even in the electronic format. This effectively prevents developing any generic mechanisms for storing, communicating and analysing the medical data, as each solution is tied up to a specific format of the data that is used by a particular organisation.

Universal Smart Patient Health Record (USPHR) represents a first attempt of a unified mechanism to create and manage electronic patient records for individual patients in a secure and transparent way, centralising a golden source of health information. It will overcome the current challenges arising from managing different formats of the diverse electronic patient records present in different institutions/scenarios by creating a universal format that will be applicable in a variety of situations. We have the following requirements for the format of USPHR:

\begin{itemize}
\item it needs to be \emph{general} enough so that it can encapsulate different use cases in the \textbf{Serums} project and wider;
  \item it needs to be \emph{specific} enough that it allows generation of machine-readable metadata that can be used for data fabrication and other parts of the \textbf{Serums} infrastructure;
  \item it needs to be \emph{flexible} enough to allow different levels of access by different interested parties, as well as the usage of blockchain to record lineage and provenance of the data;
\end{itemize}

\noindent
The USPHR will provide a fundamental core structure for the mechanisms from WP3, WP4, and WP5.
