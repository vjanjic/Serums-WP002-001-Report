\addtocounter{chapter}{1}
\setcounter{section}{0}
\setcounter{figure}{0}
\chapter*{\thechapter. Introduction}
\addcontentsline{toc}{section}{Introduction}
\label{chap:introduction}

% The \RePhrase{} consortium 
% % has evolved from a 
% % cross-disciplinary collaboration among European partners into a unique amalgam of 
% % areas of expertise over the past decade. It is comprised of 
% comprises five commercial 
% companies and three universities, and provides a service for an amalgam of highly-literate 
% computer users, the majority with doctoral degrees.

% It is straightforward to think of the \RePhrase{} consortium as a virtual organisation, 
% structured around information, rather than rules and regulations, control, and power. 
% Virtual organisations can be simply conceived as ways of structuring, managing and operating. 
% Furthermore, when widely disseminated, information appears to be instrumental in the 
% effectiveness of a virtual organisation.

The core collaboration structures for the \Serums{} project are the
internal repositories for documents and software (described in this
deliverable) and the \Serums{} web portal (which will be described in
D8.2). The repositories will hold all of the documents that will
be used for the dissemination of the project, including papers,
presentations, posters, deliverables and management reports, together
with the documents that will be used internally by the project partners,
including minutes from the meetings, project reviews and software
documentation. They will also be used to hold the software that will
be developed by the project.
Since the \Serums{} community is a geographically-dispersed
group, the repositories play a key role in binding it
together. They help to bring identity to the internal project team and
foster the integration of the \Serums{} virtual organisation since. The
repositories are the main tools for internal collaboration between
project partners, as well as serving for external dissemination by
feeding the public information (such as deliverables, presentations and
papers) to the project web portal.


The objectives of the repositories are:
\begin{itemize}
\item to enable collaboration, exchange of ideas and sharing of the results among the project participants;
\item to keep track of progress on tasks and deliverables;
\item to be the main information channel for the participants.
\end{itemize}

