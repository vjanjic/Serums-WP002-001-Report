\chapter{Introduction}
\label{chap:introduction}

\emph{Smart Patient Record} represents a central core for the information about a particular patient in the smart health centre system. It represents the central point where all the information about the patient is held, including both \emph{static} personal data such as age, height, weight, date of birth etc. that is unlikely to change, and \emph{dynamic} data related to e.g.~the treatments the patient is undertaking, screenings and prescribed medications. The main challenge for the \textbf{Serums} project is to deal with \emph{decentralisation} of the information related to a particular patient. In modern health-care systems, the data about a single patient may reside on different subsystems in the same health-care centre, or even, in some scenarios, scattered across different health-care institutions. It will also be collected from a variety of devices, some of which (e.g.~personal monitoring devices) will need to communicate with the health-care systems over open networks. We need to be able to represent all this data in a standardised way, taking into account the possible geographical distribution of both where the data is stored and where is it collected from.

It is essential to derive a standard, precise and machine readable format of the data related to a single patient. This also includes all the meta-data associated with the data (e.g.~whether the data is local or remote, how it is accessed etc). The smart patient record format also need to provide possibility of different types of access rights for different entities (patients, general practitioners, specialists) who will be accessing the data. Ideally, we want to derive a format that will cover different use cases used in the \textbf{Serums} project. This would allow generic distributed data analytics mechanism (which is the objective of WP3) to be performed on the medical data. Additionally, it would allow successful data fabrication (the objective of WP4), giving us access to a vast amount of \emph{realistic} data that will have the same format as the real data (in terms of the fields used, ranges and distribution of values in the fields and correlation between different fields), but will be synthetic, without the possibility of being associated with any real data, and therefore not being subject of privacy and ownership concerns. This data will be used throughout the project both for developing new technologies and for stress-testing them on large volumes of data. Finally, the precise format of the data is essential for storage and access mechanisms.

This deliverable represents a first step towards deriving a uniform \textbf{Serums} smart patient record format. Here, we are focusing specifically on the \emph{centralised} data, where all of the patient data is available at the same place. In the subsequent deliverable, D2.3 and D2.5, we will extend this format to support distributed data. We propose the \emph{data vault} as an appropriate generic format in which the data will be kept. Data vaults are recognised in data science as a universal format that allows easier and more automatic data analytics. Our ultimate goal is to be able to represent data of all use cases (described in Section~\ref{chap:usecase} in form of data vaults. Here, we describe the initial version of the proposed format. We first briefly describe the use cases that will be developed over the course of the \textbf{Serums} project (Section~\ref{sec:execsum}). We then discuss the universal data-vault format in Section~\ref{sec:datavault}, followed by the description of the specific format of the Edinburgh Cancer Data Gateway use case and the ways in which this can be converted into the data-vault format. Finally, we provide more information about the implementation of the universal data format in a form that can be used by the other parts of the \textbf{Serums} infrastructure. 

