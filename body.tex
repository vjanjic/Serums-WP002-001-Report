\addtocounter{chapter}{1}
\setcounter{section}{0}
\setcounter{figure}{0}
\chapter*{\thechapter. Documents and Software Repositories}
\addcontentsline{toc}{section}{Documents and Software Repositories}
\section{Documents Repository}

The documents repository is hosted using SVN, and can be accessed on

\url{bigmill.cs.st-andrews.ac.uk/local/svn/serums}.
\noindent
All project participants have
been granted access to this server. The overall organisation of the
repository, starting from the directories at the top level, is shown 
in the table below

%\begin{table}
\begin{tabular}{|p{2.5cm}|p{8.9cm}|}
\hline
\textbf{Directory} & \textbf{Content} \\
\hline \hline
\emph{Deliverables} & \Serums{} deliverables, with the following subdirectories: \emph{DX.X} for deliverable X.X, \emph{Template} for the deliverable template, \emph{PeriodX} for deliverables that belong to project period X \\
%\hline
%\emph{Documentation} & Documentation about \RePhrase{} tools \\
\hline
\emph{FactSheet} & \Serums{} fact sheet \\
\hline \emph{Logos} & Logos of the project and the institutions \\
\hline \emph{Management} & Documents related to the management of the project,
including budget, periodic reports, review letters, IPR and grant agreement \\
\hline \emph{Media} & Media files related to the project, including photos and videos \\
\hline \emph{Meetings} & Material related to project meetings (e.g.~minutes) \\
\hline \emph{Papers} & Papers produced during the project, divided into
accepted ones (in the \emph{Accepted} subdirectory) and draft ones (in the
\emph{Draft} subdirectory) \\
\hline \emph{Poster} & \Serums{} posters \\
\hline \emph{Proposal} & Grant proposal documents, including all the DoWs \\
\hline \emph{QRCode} & QR code \\
\hline \emph{Reviews} & Material related to project reviews \\
\hline
\end{tabular}

The most commonly used commands for managing a repository are
\begin{itemize}
\item \lstinline{svn co <repository_url>} for checking out a fresh copy  of a repository. For example, the \Serums{} repository can be checked out using
SSH access by executing a command

\begin{lstlisting}[basicstyle=\small]
svn co svn+ssh://<username>@bigmill.cs.st-andrews.ac.uk/local/svn/serums
\end{lstlisting}

\item \lstinline{svn up} for updating the repository and fetching changes
\item \lstinline{svn commit} for committing changes to the repository
\item \lstinline{svn add <files>} for adding files/directories to the repository
\item \lstinline{svn rm <files>} for deleting files from the repository. Deleted files will not appear in next revision.
\end{itemize}
%\end{table}

\section{Software Repository}

A separate repositorty 
for software products that will be developed over the
course of the project has been created on GitHub. This allows collaboration
between project partners on the development of tools, as well as keeping
separate private software branches by individuals. The repository was created using the
University of St Andrews account, which enables us to have a private repository that will
not be accessible (not even in read-only form) by any unauthorised users. The address of the repository is
\url{Https://github.com/uoscompsci/Serums}.

The most useful commands for managing a software Git repository are
\begin{itemize}
\item 
\lstinline{git clone <repository_url>} for checking out a fresh copy of a repository. For example, a command to
checkout the \Serums{} repository is 

\begin{lstlisting}[basicstyle=\small]
git clone https://github.com/uoscompsci/Serums.git
\end{lstlisting}

\item \lstinline{git add <file>} for adding changes in directories and files to the next commit.

\item \lstinline{git commit} for committing changes.

\item \lstinline{git checkout <branch>} for checking out a new branch.

\item \lstinline{git fetch <remote>} for downloading changes from the remote server, but without integrating them.

\item \lstinline{git pull <remote> <branch>} for downloading changes from the remote servers and integrating them into a specified branch.

\item \lstinline{git push <remote> <branch>} for pushing local changes to the remote server into a specified branch.

\end{itemize}
